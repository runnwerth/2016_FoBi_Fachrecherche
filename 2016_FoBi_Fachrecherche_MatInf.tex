\documentclass[absolute, overlay]{TIBbeamer}

%%%% Usepackages

\usepackage[english]{babel}
\usepackage{color}
\usepackage{epstopdf}
\usepackage{graphicx}
\usepackage{hyperref}
\usepackage{tikz}
\usepackage{xcolor}

%%%% Miscellaneous Settings

\graphicspath{{graphics/}}

\defbeamertemplate{description item}{align left}{\insertdescriptionitem\hfill}

\AtBeginSection{\frame{\sectionpage}}

%%%% Title Page

\title{}

\author{Mila Runnwerth}

\date{}

\begin{document}

%%%%%%%%%%%%%%%%%%%%%%%%%%%%%%%%%%%%%

\begin{frame}
\titlepage
\end{frame}

%%%%%%%%%%%%%%%%%%%%%%%%%%%%%%%%%%%%%

\begin{frame}{Inhalts\"ubersicht}
\thispagestyle{empty}
\tableofcontents
\end{frame}

%%%%%%%%%%%%%%%%%%%%%%%%%%%%%%%%%%%%%

\section{Lern- und Arbeitsverhalten}

\subsection{Studentisches Arbeiten}

%%%%%%%%%%%%%%%%%%%%%%%%%%%%%%%%%%%%%

\begin{frame}{Studentisches Arbeiten}

Als Grundlage w\"ahle ich den Prozess, der von 101 Innovations vorgeschlagen wird: \\

Bild

\end{frame}

%%%%%%%%%%%%%%%%%%%%%%%%%%%%%%%%%%%%%

\subsection{Wissenschaftliches Arbeiten}

%%%%%%%%%%%%%%%%%%%%%%%%%%%%%%%%%%%%%

\begin{frame}{Wissenschaftliches Arbeiten}

Im Studium wird die Grundlage zum wissenschaftlichen Arbeiten bereits gelegt. Allerdings kommen nun neue Randbedingungen hinzu. Als Grundlage hierf\"ur w\"ahle ich den Prozess, der im CoScience-Handbuch (Link) vorgeschlagen wird: \\

Bild

\end{frame}

%%%%%%%%%%%%%%%%%%%%%%%%%%%%%%%%%%%%%

\begin{frame}{Dann kann es los gehen.}

Nun betrachten wir den Werkzeugkasten, der in der Mathematik und in der Informatik jeweils in den entsprechenden Phasen zur Verf\"ugung steht.

\end{frame}

%%%%%%%%%%%%%%%%%%%%%%%%%%%%%%%%%%%%%

\section{Informatik}

%%%%%%%%%%%%%%%%%%%%%%%%%%%%%%%%%%%%%

\section{Mathematik}

%%%%%%%%%%%%%%%%%%%%%%%%%%%%%%%%%%%%%

\section{Fachinformationsdienst Mathematik}

%%%%%%%%%%%%%%%%%%%%%%%%%%%%%%%%%%%%%

\section{Anregungen \& Fragen}

%%%%%%%%%%%%%%%%%%%%%%%%%%%%%%%%%%%%%

\begin{frame}{Anregungen \& Fragen}

\end{frame}

%%%%%%%%%%%%%%%%%%%%%%%%%%%%%%%%%%%%%

\end{document}
